%% Documentclass:
\documentclass[OpenMind]{stjour}

%% Or,

%% Manuscript, for double spaced, larger fonts
% \documentclass[manuscript]{stjour}
%% Only needed if you use `manuscript' option
% \journalname{Open Mind}

%%%%%%%%%%%%%%%%%%%%%%%%%%%%%%%%%%%%%%%%%%%%%%%%%%
%% For production only, not authors:
%%\documentclass[OpenMind,finalfonts]{stjour}

%%%%%%%%%%% Please supply information %%%%%%%%%%%%%%%%%%%%%%%%%

%% For Open Mind:
%% Supplementary Materials:
\supplementslinks{dx.doi.org/10.1098/rsif.2013.0969}

%% If no conflicts, this command doesn't need to be used
%% \conflictsofinterest{}

%%%%%%%%%%% to be supplied by MIT Press, only %%%%%%%%%%%%%%%%%

\citation{Niyogi, R. K., Breton, Y.-A., Solomon,
R. B., Conover, K.,\\ Shizgal, P., Dayan, P. (2015).\\ 
Optimal indolence: a normative microscopic approach to work and leisure. Open Mind 1(1):
1−12.}

\received{20 October 2013}
\accepted{7 November 2013}
\published{26 January 2014}

%% DOI address:
\setdoi{10.1098/rsif.2013.0969}

%%%%%%%% End MIT Press commands %%%%%%%%%%

%%%%%%%%%%%%%%%%%%%%%%%%%%%%%%%%%%%%%%%%%%%%%%%%%%%%%%%%%%%%%%%
%% author definitions should be placed here:

%% example definition
\def\taupav{\tau_{\mathrm{Pav}}}
\begin{document}

\title{Title of Article}
\subtitle{Subtitle Here}

%% If shortened title for running head is needed so that the article title can fit
%%   in the running head, use [] argument, ie,
%%
%%   \title[Shortened Title for Running Head]{Title of Article}
%%   \subtitle{Subtitle Here}

%% Since we use \affil{} in the argument of the \author command, we
%% need to supply another version of the author names, without \affil{}
%% to be used for running heads:

\author[Author Names]
{Author Names with affiliations\thanks{Current address: MacMurdo
Sound, Antartica}\affil{1},
Another Name\affil{2}, Still another Name\affil{2},\\
\and Final Name\affil{1}}

\affiliation{1}{Department, Institution, City, Country}

%ie.
%\affiliation{1}{Gatsby Computational Neuroscience Unit, University
%College London, London, United Kingdom} 

\affiliation{2}{Another Department, Institution, City, Country}

%ie
%\affiliation{2}{Center for Studies in
%Behavioral Neurobiology, Concordia University, Montreal, Quebec,
%Canada}

\correspondingauthor{Author Name}{Corresponding author email address}

% ie,
%\correspondingauthor{Ritwik K. Niyogi}{ritwik.niyogi@gatsby.ucl.ac.uk}

\keywords{(a series of uncapitalized words, separated with commas)}

%ie
%\keywords{work, leisure, normative, microscopic,  reinforcement learning, economics}

\begin{abstract}
Abstract text here.
\end{abstract}


\section{Sample Section}
Text here. Text here. Text here. Text here.
Text here. Text here. Text here. Text here.
Text here. Text here. Text here. Text here.
Text here. Text here. Text here. Text here.

\subsection{Sample Subsection}
Text here. Text here. Text here. Text here.
Text here. Text here. Text here. Text here.
Text here. Text here. Text here. Text here.
Text here. Text here. Text here. Text here.

\subsubsection{Sample Subsubsection}
Text here. Text here. Text here. Text here.
Text here. Text here. Text here. Text here.
Text here. Text here. Text here. Text here.
Text here. Text here. Text here. Text here.

\section{Sample equations}
\begin{equation}
\label{eq:rhoCHT}
\rho^{\pi}= \frac{RI + \mathbb{E}_{\pi([L,\tau_L]|\textrm{post})}
\left[C_L(\taupav+\tau_L) \right]   +
\displaystyle{\int_{0}^{P}}{dw~ \mathbb{E}_{\pi_{w_L}}}
\Biggl[\/\sum_{n_{L|[\textrm{pre},w]}}C_L(\tau_L)
\Biggr]            }      {P +
\mathbb{E}_{\pi([L,\tau_L] |\textrm{post})}[\tau_{L}] +\taupav +
\displaystyle{ \int_{0}^{P}}{dw~ \mathbb{E}_{\pi_{w_L}}}   
\Biggl[\sum_{n_{L|[\textrm{pre},w]}}\tau_L\Biggr]  
}
\end{equation}
As long as
$RI - K_LP > 
\frac{1}{\beta}$
\begin{equation}
%\def\theequation{5.1}
\left.\begin{array}{lrcl}
&\rho^{\pi} &=&  \displaystyle\frac{\beta ( RI + K_L \taupav )-1} {\beta
(P+\taupav )}    \\[12pt]
\hbox{and}\hbox to .25in{\hfill}&\mathbb{E}[\tau_L | \text{post}] &=&\displaystyle \frac{P+\taupav}{\beta ( RI -
K_LP)-1}  
\label{eq:analytical_linear}
\end{array}\right\}\hbox to 1.25in{\hfill}
\end{equation} 

\subsection{Simple code sample}

\begin{code}
\begin{verbatim}
procedure bubbleSort( A : list of sortable items )
    n = length(A)
    repeat
       newn = 0
       for i = 1 to n-1 inclusive do
          if A[i-1] > A[i] then
             swap(A[i-1], A[i])
             newn = i
          end if
       end for
       n = newn
    until n = 0
end procedure
\end{verbatim}
\end{code}



\subsection{Algorithm environment}

%% \begin{algorithm} takes option [p][b][t][h],  or some combination, like \begin{figure}
%% See documentation for algorithmic.sty for more information on formatting algorithms.

\begin{algorithm}[h]
\caption{A sample in an algorithm environment.}
\begin{algorithmic}
\If {$i\geq maxval$}
    \State $i\gets 0$
\Else
    \If {$i+k\leq maxval$}
        \State $i\gets i+k$
    \EndIf
\EndIf
\end{algorithmic}
\end{algorithm}


\section{Jargon definitions}

\begin{wideboxedtext}
\begin{glossary}
\symbol{Term} Definition
\symbol{Term} Definition
%ie
%\symbol{$1/\lambda$} mean of exponential effective prior probability density for leisure time
%\symbol{CHT} Cumulative Handling Time 
\end{glossary}
\end{wideboxedtext}

\begin{boxedtext}
\begin{glossary}
\symbol{Term} Definition
\symbol{Term} Definition
%ie
%\symbol{$1/\lambda$} mean of exponential effective prior probability density for leisure time
%\symbol{CHT} Cumulative Handling Time 
\end{glossary}
\end{boxedtext}

\section{Itemized Lists}

\subsection{Roman list:}

\begin{enumerate}
\item[(i)] at high 
payoffs, subjects work almost continuously, engaging in little leisure
in between work bouts; 
\item[(ii)] at low payoffs, they 
engage in leisure all at once, in long bouts after working, rather
than distributing the same amount of leisure time into multiple short
leisure bouts; 
\item[(iii)] subjects work continuously for the entire price duration, as long as
the price is not very long (as shown by an analysis conducted by Y-AB, to be published separately); %(\textbf{Figure \ref{fig:task_data}D}).
\item[(iv)] the duration of leisure bouts is variable.
\end{enumerate}

\subsection{Numbered list:}

\begin{enumerate}
\item at high 
payoffs, subjects work almost continuously, engaging in little leisure
inbetween work bouts; 
\item at low payoffs, they 
engage in leisure all at once, in long bouts after working, rather
than distributing the same amount of leisure time into multiple short
leisure bouts; 
\item subjects work continuously for the entire price duration, as long as
the price is not very long (as shown by an analysis conducted by Y-AB, to be published separately); %(\textbf{Figure \ref{fig:task_data}D}).
\item the duration of leisure bouts is variable.
\end{enumerate}


\subsection{Bulleted list:}

\begin{itemize}
\item at high 
payoffs, subjects work almost continuously, engaging in little leisure
inbetween work bouts; 
\item at low payoffs, they 
engage in leisure all at once, in long bouts after working, rather
than distributing the same amount of leisure time into multiple short
leisure bouts; 
\item subjects work continuously for the entire price duration, as long as
the price is not very long (as shown by an analysis conducted by Y-AB, to be published separately); %(\textbf{Figure \ref{fig:task_data}D}).
\item the duration of leisure bouts is variable.
\end{itemize}


\section{Sample citations}
For general information on the correct form for citations using
the APA 6 format, see the following sites:
\href{https://owl.english.purdue.edu/owl/resource/560/02/}
{APA 6, In-text citations, The Basics} and
\href{https://owl.english.purdue.edu/owl/resource/560/03/}
{APA 6, In-text citations}

\section{Natbib citation mark up}

\subsection{Single citations}
\noindent
\begin{tabular}{ll}
\bf Type&\bf Results\\
\hline
\verb+\citet{jon90}+&Jones et al. (1990)\\
\verb+\citet[chap. 2]{jon90}+&Jones et al. (1990, chap. 2)\\
    \verb+\citep{jon90}+	    &   	(Jones et al., 1990)\\
    \verb+\citep[chap. 2]{jon90}+ 	&    	(Jones et al., 1990, chap. 2)\\
    \verb+\citep[see][]{jon90}+ 	 &    	(see Jones et al., 1990)\\
    \verb+\citep[see][chap. 2]{jon90}+ 	&    	(see Jones et al., 1990, chap. 2)\\
    \verb+\citet*{jon90}+ 	    &    	Jones, Baker, and Williams (1990)\\
    \verb+\citep*{jon90}+	    &    	(Jones, Baker, and Williams,
    1990) \\
\end{tabular}

For example, some citations from the OpenMindSample bibliography:
citet:\citet{anderson}, citep:\citep{antibayes}, and
cite*: \citet*{anderson}.

\subsection{Multiple citations}
Multiple citations may be made by including more than one citation
key in the \verb+\cite+ command argument.

\noindent
\begin{tabular}{ll}
\bf Type&\bf Results\\
\hline
\verb+\citet{jon90,jam91}+&Jones et al. (1990); James et al. (1991)\\
\verb+\citep{jon90,jam91}+&(Jones et al., 1990; James et al. 1991)\\
\verb+\citep{jon90,jon91}+&(Jones et al., 1990, 1991)\\
\verb+\citep{jon90a,jon90b}+&(Jones et al., 1990a,b)\\
\end{tabular}

For example, multiple citations from the OpenMindSample bibliography:
citet:\citet{anderson,antibayes}, citep:\citep{anderson,antibayes}.
As you see, the citations are automatically hyperlinked to their
reference in the bibliography.

\newpage

\section{Sample figures}

\begin{figure}[h] 
\centerline{\includegraphics[width=\textwidth]{Fig1}}
\caption{(Colour online) \textbf{Task and key features of the
 data.} \\
 A) Cumulative handling time (CHT) task. Grey bars denote work
(depressing a lever), white gaps show leisure. The subject must
 accumulate work up to a total period of time called the
\emph{price} ($P$) in order to obtain a single reward (black dot) of subjective reward
intensity $RI$. The trial duration is $25\times \mathrm{price}$ (plus
$2$s each time the price is attained, during which the lever is retracted so it cannot
work; not shown).
}
\label{fig:task_data}
\end{figure}

\begin{figure}[ht] 
\widefigure{\fullpagewidth}{Fig1}
\caption{(Colour online) \textbf{Task and key features of the
 data.} \\
A) Cumulative handling time (CHT) task. Grey bars denote work
(depressing a lever), white gaps show leisure. The subject must
accumulate work up to a total period of time called the
\emph{price} ($P$) in order to obtain a single reward (black dot) of subjective reward
intensity $RI$. The trial duration is $25\times \mathrm{price}$ (plus
$2$s each time the price is attained, during which the lever is retracted so it cannot
work; not shown).
}
\label{newfig:task_data}
\end{figure}

\clearpage
\section{Sample tables}

\begin{table}[!ht]
\caption{Time of the Transition Between Phase 1 and Phase 2$^{a}$}
\label{tab:label}
\centering
\begin{tabular}{lc}
\hline
 Run  & Time (min)  \\
\hline
  $l1$  & 260   \\
  $l2$  & 300   \\
  $l3$  & 340   \\
  $h1$  & 270   \\
  $h2$  & 250   \\
  $h3$  & 380   \\
  $r1$  & 370   \\
  $r2$  & 390   \\
\hline
\multicolumn{2}{l}{$^{a}$Table note text here.}
\end{tabular}
\end{table}

\begin{table}[ht]
\widecaption{Sample table taken from [treu03]\label{tbl-1}}
\begin{widetable}
\advance\tabcolsep-1pt
\small
\begin{tabular}{ccrrccccccccc}
\hline
\bf 
POS &\bf  chip &\multicolumn1c{\bf ID} &\multicolumn1c{\bf X}
&\multicolumn1c{\bf Y} &\bf
RA &\bf DEC &\bf IAU$\pm$ $\delta$ IAU &\bf
IAP1$\pm$ $\delta$ IAP1 &\bf IAP2 $\pm$ $\delta$
IAP2 &\bf star &\bf E &\bf Comment\\
\hline
0 & 2 & 1 & 1370.99 & 57.35\rlap{$^a$}    &   6.651120 &  17.131149 &
21.344$\pm$0.006\rlap{$^b$}  & 2 4.385$\pm$0.016 & 23.528$\pm$0.013 & 0.0 & 9 & -    \\
0 & 2 & 2 & 1476.62 & 8.03     &   6.651480 &  17.129572 & 21.641$\pm$0.005  & 2 3.141$\pm$0.007 & 22.007$\pm$0.004 & 0.0 & 9 & -    \\
0 & 2 & 3 & 1079.62 & 28.92    &   6.652430 &  17.135000 & 23.953$\pm$0.030  & 2 4.890$\pm$0.023 & 24.240$\pm$0.023 & 0.0 & - & -    \\
0 & 2 & 4 & 114.58  & 21.22    &   6.655560 &  17.148020 & 23.801$\pm$0.025  & 2 5.039$\pm$0.026 & 24.112$\pm$0.021 & 0.0 & - & -    \\
0 & 2 & 5 & 46.78   & 19.46    &   6.655800 &  17.148932 & 23.012$\pm$0.012  & 2 3.924$\pm$0.012 & 23.282$\pm$0.011 & 0.0 & - & -    \\
0 & 2 & 6 & 1441.84 & 16.16    &   6.651480 &  17.130072 & 24.393$\pm$0.045  & 2 6.099$\pm$0.062 & 25.119$\pm$0.049 & 0.0 & - & -    \\
0 & 2 & 7 & 205.43  & 3.96     &   6.655520 &  17.146742 & 24.424$\pm$0.032  & 2 5.028$\pm$0.025 & 24.597$\pm$0.027 & 0.0 & - & -    \\
0 & 2 & 8 & 1321.63 & 9.76     &   6.651950 &  17.131672 &
22.189$\pm$0.011  & 2 4.743$\pm$0.021 & 23.298$\pm$0.011 & 0.0 & 4 &
edge \\
\hline
\multicolumn{13}{l}{%
Table 2 is published in its entirety in the electronic
edition of the {\it Astrophysical Journal}.}\\[3pt]
\multicolumn{13}{l}{%
$^a$ Sample footnote for table 2.}\\[3pt]
\multicolumn{13}{l}{%
$^b$ Another sample footnote for table 2.}
\end{tabular}
\end{widetable}
\end{table}

\begin{table}[p]
\rotatebox{90}{\vbox{\hsize=\textheight
\caption{Here is a caption for a table that is found in landscape
mode.}
\begin{tabular}{ccrrccccccccc}
\hline
\bf 
POS &\bf  chip &\multicolumn1c{\bf ID} &\multicolumn1c{\bf X}
&\multicolumn1c{\bf Y} &\bf
RA &\bf DEC &\bf IAU$\pm$ $\delta$ IAU &\bf
IAP1$\pm$ $\delta$ IAP1 &\bf IAP2 $\pm$ $\delta$
IAP2 &\bf star &\bf E &\bf Comment\\
\hline
0 & 2 & 1 & 1370.99 & 57.35\rlap{$^a$}    &   6.651120 &  17.131149 &
21.344$\pm$0.006\rlap{$^b$}  & 2 4.385$\pm$0.016 & 23.528$\pm$0.013 & 0.0 & 9 & -    \\
0 & 2 & 2 & 1476.62 & 8.03     &   6.651480 &  17.129572 & 21.641$\pm$0.005  & 2 3.141$\pm$0.007 & 22.007$\pm$0.004 & 0.0 & 9 & -    \\
0 & 2 & 3 & 1079.62 & 28.92    &   6.652430 &  17.135000 & 23.953$\pm$0.030  & 2 4.890$\pm$0.023 & 24.240$\pm$0.023 & 0.0 & - & -    \\
0 & 2 & 4 & 114.58  & 21.22    &   6.655560 &  17.148020 & 23.801$\pm$0.025  & 2 5.039$\pm$0.026 & 24.112$\pm$0.021 & 0.0 & - & -    \\
0 & 2 & 5 & 46.78   & 19.46    &   6.655800 &  17.148932 & 23.012$\pm$0.012  & 2 3.924$\pm$0.012 & 23.282$\pm$0.011 & 0.0 & - & -    \\
0 & 2 & 6 & 1441.84 & 16.16    &   6.651480 &  17.130072 & 24.393$\pm$0.045  & 2 6.099$\pm$0.062 & 25.119$\pm$0.049 & 0.0 & - & -    \\
0 & 2 & 7 & 205.43  & 3.96     &   6.655520 &  17.146742 & 24.424$\pm$0.032  & 2 5.028$\pm$0.025 & 24.597$\pm$0.027 & 0.0 & - & -    \\
0 & 2 & 8 & 1321.63 & 9.76     &   6.651950 &  17.131672 &
22.189$\pm$0.011  & 2 4.743$\pm$0.021 & 23.298$\pm$0.011 & 0.0 & 4 &
edge \\
\hline
\multicolumn{13}{l}{%
Table 2 is published in its entirety in the electronic
edition of the {\it Astrophysical Journal}.}\\[3pt]
\multicolumn{13}{l}{%
$^a$ Sample footnote for table 2.}\\[3pt]
\multicolumn{13}{l}{%
$^b$ Another sample footnote for table 2.}
\end{tabular}
}}
\end{table}
\clearpage


\vglue 3in
Example of table continuing over pages:


\begin{center}
\begin{longtable}{ccc@{}}
\caption{ApJ costs from 1991 to 2013
\label{tab:table}} \\[2pt]
\hline
\bf Year & \bf Subscription & \bf Publication \\
 & \bf cost &\bf charges\\
 & \bf(\$) & \bf (\$/page)\\
\hline
\endfirsthead

\multicolumn3c{Table \thetable, \it continued from previous page.}\\[6pt]
\multicolumn3c{ApJ costs from 1991 to 2013}\\[2pt]
\hline
\bf Year & \bf Subscription & \bf Publication \\
 & \bf cost &\bf charges\\
 & \bf(\$) & \bf (\$/page)\\
\hline
\endhead
\\\hline
\\[-8pt]
\multicolumn{3}{r}{\it Table continued on next page}\\ 
\endfoot

\hline
\endlastfoot

1991 & 600 & 100 \\
1992 & 650 & 105 \\
1993 & 550 & 103 \\
1994 & 450 & 110 \\
1995 & 410 & 112 \\
1996 & 400 & 114 \\
1997 & 525 & 115 \\
1998 & 590 & 116 \\
1999 & 575 & 115 \\
2000 & 450 & 103 \\
2001 & 490 &  90 \\
2002 & 500 &  88 \\
2003 & 450 &  90 \\
2004 & 460 &  88 \\
2005 & 440 &  79 \\
2006 & 350 &  77 \\
2007 & 325 &  70 \\
2008 & 320 &  65 \\
2009 & 190 &  68 \\
2010 & 280 &  70 \\
2011 & 275 &  68 \\
2012 & 150 &  56 \\
2013 & 140 &  55 \\
\end{longtable}
\end{center}

\section{Supportive Information}
Here you enter further sources of information, if desired.

%% A possible entry might be:
% No supportive information is available at this time.


\acknowledgments
Enter your acknowledgments here.

%% ie.,

% The authors thank Laurence Aitchison for fruitful discussions.  RKN
% and PD received funding from the Gatsby Charitable Foundation. Y-AB,
% RBS, KC and PS received funding from Canadian Institutes of Health
% Research grant $MOP74577$,
% Fond de recherche Qu\'{e}bec - Sant\'{e} (Group grant to the Groupe 
% de recherche en neurobiologie comportementale, Shimon Amir, P.I.), and
% Concordia University Research Chair (Tier I). 

\authorcontributions 
Who helped formulate the project, who supplied data, analyses and
experiments, etc.

%% ie.
%% Project was formulated by RKN, PD, PS,
%% based on substantial data, analyses and experiments of Y-AB, KC, RS,
%% PS. RKN, PD formalised the model, RKN implemented and ran the model;
%% RKN analysed the molecular ethogram data; Y-AB formalised and
%% implemented a CTMC model. All authors wrote the manuscript.


%%%%%%%%%%%%%%%%%%%%%%%
%% The bibliography

%% \nocite{*} is used here as a quick way to get every entry  in the .bib file to
%% appear in the bibliography. Normally the bibliography is made using only
%% the bibentries that you cite using \cite{}, or one of the Natbib citation
%% entries, like \citep{}, \citet{} etc.

\nocite{*}
\bibliography{bibsamp}


\appendix

\section{Sample Appendix Section}
We derive the result in Eq. \eqref{eq:analytical_linear}. We consider a linear $C_L(\tau_L+\taupav)=K_L(\tau_L+\taupav)$, and
make two further 
simplifications: (i) the subject does not
engage in leisure in the pre-reward state (and so works for the whole
price when it works); and (ii) \emph{a priori}, arbitrarily long leisure durations are possible
($\lambda=0$).
Then the reward rate in Eq. \eqref{eq:rhoCHT} becomes
\begin{equation}\label{eq:analyticalrho}
\rho^{\pi}= \frac{\vrule height 10pt width0pt RI + K_L \{~ \mathbb{E}[ \tau_L | \textrm{post} ] + \taupav  \} }
     {P +
\mathbb{E}[ \tau_L | \textrm{post} ] +\taupav  } 
\end{equation} 
As discussed in the \emph{Results} section, the probability of engaging in instrumental leisure in the post-reward state is $\pi([L,\tau_L]
~| \textrm{post}) = \exp\left[-\{\beta (\rho^\pi-K_L) 
 \} \tau_L\right]$, which is an exponential distribution with
mean 
\begin{equation}
\mathbb{E}[\tau_L | \textrm{post}]=\frac{1}{\beta (\rho^\pi-K_L) }
\label{eq:analyticaltauL}
\end{equation}  
Re-arranging terms of this equation,
\begin{equation}\label{eq:rhoversion2}
\rho^{\pi}=\frac{1}{\beta ~\mathbb{E}[\tau_L | \textrm{post}]} +K_L 
\end{equation} 
Equating Eqs. \eqref{eq:analyticalrho} and \eqref{eq:rhoversion2} and solving for the mean instrumental leisure duration $\mathbb{E}[\tau_L | \textrm{post}]$, we derive
\begin{equation}
\mathbb{E}[\tau_L | \textrm{post}] = \frac{P+\taupav}{\beta ( RI - K_LP)-1} 
\label{eq:solvedtauL}
\end{equation}
which is the second line of Eq.\eqref{eq:analytical_linear}. This is the mean instrumental leisure duration as long as  $RI - K_LP>1$, and  $\mathbb{E}[\tau_L | \textrm{post}] \rightarrow \infty$ otherwise. When the former condition holds, we may
substitute Eq. \eqref{eq:solvedtauL} into Eq. \eqref{eq:analyticalrho} and solve for $\rho^{\pi}$

\begin{figure}
\widefigure{\fullpagewidth}{Fig1}
\caption{Sample Appendix Caption. Here is a caption that might
appear in an appendix. It is as wide as the full width of the page.}
\end{figure}

\newpage

\section{Making Your Bibliography for an Open Mind Article}
{\it Open Mind} uses the APA author-date  bibliography style,
apacite.bst. For more
information on apacite, for examples in how to make your .bib file and more, see:\\
\href{http://mirror.jmu.edu/pub/CTAN/biblio/bibtex/contrib/apacite/apacite.pdf}
{http://mirror.jmu.edu/pub/CTAN/biblio/bibtex/contrib/apacite/apacite.pdf}

\noindent
(In spite of the mention of apacite cite commands, please use only
Natbib commands for in text citations, as shown above.)

\subsection{BibTeX}
You will need to use BibTeX to form your bibliography; typing in the
references would be
a huge and unpleasant task. Look at the openmindsample.bbl file and you'll see why
typing in the bibitems would be difficult. 

For a good basic introduction to using BibTeX, see
\href{https://www.economics.utoronto.ca/osborne/latex/BIBTEX.HTM}
{https://www.economics.utoronto.ca/osborne/latex/BIBTEX.HTM}

When you use BibTeX, the form of the bibliography will be correct. You
don't need to supply a bibliography style, since that is built into
the stjour.cls file.

\end{document}

