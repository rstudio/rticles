%% Version 6.1, 1 September 2021
%
%%%%%%%%%%%%%%%%%%%%%%%%%%%%%%%%%%%%%%%%%%%%%%%%%%%%%%%%%%%%%%%%%%%%%%
% TemplateV6.1.tex --  LaTeX-based blank template for submissions to the
% American Meteorological Society
%
%%%%%%%%%%%%%%%%%%%%%%%%%%%%%%%%%%%%%%%%%%%%%%%%%%%%%%%%%%%%%%%%%%%%%
% PREAMBLE
%%%%%%%%%%%%%%%%%%%%%%%%%%%%%%%%%%%%%%%%%%%%%%%%%%%%%%%%%%%%%%%%%%%%%

%% Start with one of the following:
% 1.5-SPACED VERSION FOR SUBMISSION TO THE AMS
\documentclass[$if(twocol)$twocol$endif$]{ametsocV6.1}

% TWO-COLUMN JOURNAL PAGE LAYOUT---FOR AUTHOR USE ONLY
% \documentclass[twocol]{ametsocV6.1}

%%%%%%%%%%%%%%%%%%%%%%%%%%%%%%%%

$for(header-includes)$
$header-includes$
$endfor$

%%% To be entered by author:

%% May use \\ to break lines in title:

\title{$title$}

%% Enter authors' names and affiliations as you see in the examples below.
%
%% Use \correspondingauthor{} and \thanks{} (\thanks command to be used for affiliations footnotes,
%% such as current affiliation, additional affiliation, deceased, co-first authors, etc.)
%% immediately following the appropriate author.
%
%% Note that the \correspondingauthor{} command is NECESSARY.
%% The \thanks{} commands are OPTIONAL.
%
%% Enter affiliations within the \affiliation{} field. Use \aff{#} to indicate the affiliation letter at both the
%% affiliation and at each author's name. Use \\ to insert line breaks to place each affiliation on its own line.

%\authors{Author One,\aff{a}\correspondingauthor{Author One, email@email.com}
%Author Two,\aff{a}
%Author Three,\aff{b}
%Author Four,\aff{a}
%Author Five\thanks{Author Five's current affiliation: NCAR, Boulder, Colorado},\aff{c}
%Author Six,\aff{c}
%Author Seven,\aff{d}
% and Author Eight\aff{a,d}
%}
%
%\affiliation{\aff{a}{First Affiliation}\\
%\aff{b}{Second Affiliation}\\
%\aff{c}{Third Affiliation}\\
%\aff{d}{Fourth Affiliation}
%}

% Credit to https://stackoverflow.com/a/67609365 for different last-author case.
\authors{
$if(authors/allbutlast)$
$for(authors/allbutlast)$
$it.name$$if(it.current)$\thanks{$it.current$}$endif$,\aff{$it.aff$}$if(it.email)$\correspondingauthor{$it.name$, $it.email$}$endif$
$endfor$
and~
$endif$
$for(authors/last)$
$it.name$$if(it.current)$\thanks{$it.current$}$endif$\aff{$it.aff$}$if(it.email)$\correspondingauthor{$it.name$, $it.email$}$endif$
$endfor$
}

% Credit to https://stackoverflow.com/a/67609365 for different last-affiliation case.
\affiliation{
$if(affiliations/allbutlast)$
$for(affiliations/allbutlast)$
\aff{$it.aff$}{$it.name$}\\
$endfor$
$endif$
$for(affiliations/last)$
\aff{$it.aff$}{$it.name$}
$endfor$
}

%%%%%%%%%%%%%%%%%%%%%%%%%%%%%%%%%%%%%%%%%%%%%%%%%%%%%%%%%%%%%%%%%%%%%
% ABSTRACT
%
% Enter your abstract here
% Abstracts should not exceed 250 words in length!
%

\abstract{$abstract$}

\begin{document}

%% Necessary!
\maketitle

\bibliographystyle{ametsocV6}
%%%%%%%%%%%%%%%%%%%%%%%%%%%%%%%%%%%%%%%%%%%%%%%%%%%%%%%%%%%%%%%%%%%%%
% SIGNIFICANCE STATEMENT/CAPSULE SUMMARY
%%%%%%%%%%%%%%%%%%%%%%%%%%%%%%%%%%%%%%%%%%%%%%%%%%%%%%%%%%%%%%%%%%%%%
%
% If you are including an optional significance statement for a journal article or a required capsule summary for BAMS
% (see www.ametsoc.org/ams/index.cfm/publications/authors/journal-and-bams-authors/formatting-and-manuscript-components for details),
% please apply the necessary command as shown below:
%
% Significance Statement (all journals except BAMS)
%
%\statement
%	 Enter significance statement here, no more than 120 words. See \url{www.ametsoc.org/index.cfm/ams/publications/author-information/significance-statements/} for details.
%

$if(statement)$
%$if(twocol)$\twocolsig$else$\statement$endif$
$statement$
$endif$ % FIXME, AMS template says to use twocolsig instead of sig for twocol, but there is no sig, is two col twocolstatement or twocolsig?

%% Capsule (BAMS only)
%%
%\capsule
%       Enter BAMS capsule here, no more than 30 words. See \url{www.ametsoc.org/index.cfm/ams/publications/author-information/formatting-and-manuscript-components/#capsule} for details.
%

$if(capsule)$
%$if(twocol)$\twocolcapsule$else$\capsule$endif$
$capsule$
$endif$

%% * * If using twocol mode, you will need to use the commands "twocolsig" and "twocolcapsule" in place of "sig" and "capsule"
%%      to ensure that the text box correctly spans across both columns.
%

$for(include-before)$
$include-before$

$endfor$

%%%%%%%%%%%%%%%%%%%%%%%%%%%%%%%%%%%%%%%%%%%%%%%%%%%%%%%%%%%%%%%%%%%%%
% MAIN BODY OF PAPER
%%%%%%%%%%%%%%%%%%%%%%%%%%%%%%%%%%%%%%%%%%%%%%%%%%%%%%%%%%%%%%%%%%%%%
%

$body$

$for(include-after)$
$include-after$

$endfor$

\end{document}
