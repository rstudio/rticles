%% josis.tex 1.4   2016-09-15    JoSIS latex template
%------------------------------------------------------------------
% Filename: josis_template.tex
%
% This file is intended as a template for typesetting articles for the
%
%                        Journal of Spatial Information Science.
%
% Please edit this template to generate your own formatted manuscripts
% for submission to JOSIS. See http://josis.org for further details.
%


%%% JOSIS checks in typesetting
%%% * All titles and sections lower case *EXCEPT short title  [ ]
%%% * Remove author postal addresses, only have geographic places and institutions [ ]
%%% * Consistent use of Section, Figure, Table (capitalized and in full) [ ]
%%% * 10 keywords (and all lower case) [ ]
%%% * Remove all avoidable footnotes [ ]
%%% * Use double quotation marks (``'' not "" or `') [ ]
%%% * Punctuation inside quotations [ ]
%%% * E.g. and i.e. followed by comma [ ]
%%% * cf. followed by tilde [ ]
%%% * Itemize and enumerate correctly punctuated [e.g., "1. x, 2. y, and 3. x." ]
%%% * And/or lists using American English punctuation (e.g., "x, y, and z") [ ]
%%% * Bibliography (e.g., en-dashes for number ranges, consistent "Proc.~" for Proceedings of..., etc.) []
%%% * Acknowledgment style use section* [ ]
%%% * et al. no italics, but with dot  [ ]
%%% * All captions end with full stop  [ ]
%%% * Table captions under, not over table  [ ]
%%% * Adjust urls with burlalt [ ]
%%% * Check correct use of hyphens, emdashes, endashes  [ ]
%%% * Perform spell check  [ ]

%%% JOSIS checks directly before publication
%%% Check DOI, page numbers on article and web site. [ ]
%%% Update web site with final title, abstract, keywords. [ ]
%%% Build with distiller for DOI links. [ ]


% Required documentclass definition for JOSIS
\documentclass{josis}
\usepackage{hyperref}
\usepackage[hyphenbreaks]{breakurl}
\usepackage{booktabs}
\usepackage{stmaryrd}
\usepackage[T1]{fontenc}
\usepackage{cite}

% Suggested packages for algorithm formatting
\usepackage{algorithm}
%\usepackage{algorithmic}
\usepackage{algpseudocode}


\usepackage[table]{xcolor}
\usepackage{amssymb,amsmath}

\renewcommand{\topfraction}{0.9}
\renewcommand{\textfraction}{0.1}

% Page setup and overhangs
\sloppy
\widowpenalty=10000
\clubpenalty=10000
\hyphenpenalty=75

% For figures and more when knitting from .Rmd:
% https://stackoverflow.com/questions/41052687/
\usepackage{color}
\usepackage{fancyvrb}
\newcommand{\VerbBar}{|}
\newcommand{\VERB}{\Verb[commandchars=\\\{\}]}
\DefineVerbatimEnvironment{Highlighting}{Verbatim}{commandchars=\\\{\}}
% Add ',fontsize=\small' for more characters per line
\usepackage{framed}
\definecolor{shadecolor}{RGB}{248,248,248}
\newenvironment{Shaded}{\begin{snugshade}}{\end{snugshade}}
\newcommand{\AlertTok}[1]{\textcolor[rgb]{0.94,0.16,0.16}{#1}}
\newcommand{\AnnotationTok}[1]{\textcolor[rgb]{0.56,0.35,0.01}{\textbf{\textit{#1}}}}
\newcommand{\AttributeTok}[1]{\textcolor[rgb]{0.77,0.63,0.00}{#1}}
\newcommand{\BaseNTok}[1]{\textcolor[rgb]{0.00,0.00,0.81}{#1}}
\newcommand{\BuiltInTok}[1]{#1}
\newcommand{\CharTok}[1]{\textcolor[rgb]{0.31,0.60,0.02}{#1}}
\newcommand{\CommentTok}[1]{\textcolor[rgb]{0.56,0.35,0.01}{\textit{#1}}}
\newcommand{\CommentVarTok}[1]{\textcolor[rgb]{0.56,0.35,0.01}{\textbf{\textit{#1}}}}
\newcommand{\ConstantTok}[1]{\textcolor[rgb]{0.00,0.00,0.00}{#1}}
\newcommand{\ControlFlowTok}[1]{\textcolor[rgb]{0.13,0.29,0.53}{\textbf{#1}}}
\newcommand{\DataTypeTok}[1]{\textcolor[rgb]{0.13,0.29,0.53}{#1}}
\newcommand{\DecValTok}[1]{\textcolor[rgb]{0.00,0.00,0.81}{#1}}
\newcommand{\DocumentationTok}[1]{\textcolor[rgb]{0.56,0.35,0.01}{\textbf{\textit{#1}}}}
\newcommand{\ErrorTok}[1]{\textcolor[rgb]{0.64,0.00,0.00}{\textbf{#1}}}
\newcommand{\ExtensionTok}[1]{#1}
\newcommand{\FloatTok}[1]{\textcolor[rgb]{0.00,0.00,0.81}{#1}}
\newcommand{\FunctionTok}[1]{\textcolor[rgb]{0.00,0.00,0.00}{#1}}
\newcommand{\ImportTok}[1]{#1}
\newcommand{\InformationTok}[1]{\textcolor[rgb]{0.56,0.35,0.01}{\textbf{\textit{#1}}}}
\newcommand{\KeywordTok}[1]{\textcolor[rgb]{0.13,0.29,0.53}{\textbf{#1}}}
\newcommand{\NormalTok}[1]{#1}
\newcommand{\OperatorTok}[1]{\textcolor[rgb]{0.81,0.36,0.00}{\textbf{#1}}}
\newcommand{\OtherTok}[1]{\textcolor[rgb]{0.56,0.35,0.01}{#1}}
\newcommand{\PreprocessorTok}[1]{\textcolor[rgb]{0.56,0.35,0.01}{\textit{#1}}}
\newcommand{\RegionMarkerTok}[1]{#1}
\newcommand{\SpecialCharTok}[1]{\textcolor[rgb]{0.00,0.00,0.00}{#1}}
\newcommand{\SpecialStringTok}[1]{\textcolor[rgb]{0.31,0.60,0.02}{#1}}
\newcommand{\StringTok}[1]{\textcolor[rgb]{0.31,0.60,0.02}{#1}}
\newcommand{\VariableTok}[1]{\textcolor[rgb]{0.00,0.00,0.00}{#1}}
\newcommand{\VerbatimStringTok}[1]{\textcolor[rgb]{0.31,0.60,0.02}{#1}}
\newcommand{\WarningTok}[1]{\textcolor[rgb]{0.56,0.35,0.01}{\textbf{\textit{#1}}}}
\providecommand{\tightlist}{%
  \setlength{\itemsep}{0pt}\setlength{\parskip}{0pt}}
\usepackage{longtable,booktabs,array}
\newlength{\cslhangindent}
\setlength{\cslhangindent}{1.5em}
\newlength{\csllabelwidth}
\setlength{\csllabelwidth}{3em}
\newenvironment{CSLReferences}[2] % #1 hanging-ident, #2 entry spacing
 {% don't indent paragraphs
  \setlength{\parindent}{0pt}
  % turn on hanging indent if param 1 is 1
  \ifodd #1 \everypar{\setlength{\hangindent}{\cslhangindent}}\ignorespaces\fi
  % set entry spacing
  \ifnum #2 > 0
  \setlength{\parskip}{#2\baselineskip}
  \fi
 }%
 {}
\usepackage{calc}
\newcommand{\CSLBlock}[1]{#1\hfill\break}
\newcommand{\CSLLeftMargin}[1]{\parbox[t]{\csllabelwidth}{#1}}
\newcommand{\CSLRightInline}[1]{\parbox[t]{\linewidth - \csllabelwidth}{#1}\break}
\newcommand{\CSLIndent}[1]{\hspace{\cslhangindent}#1}

% Article details for accepted manuscripts will be added by editorial staff
% Omit year if article in press
% Omit number if article under review
\josisdetails{%
   number=N, year=YYYY, firstpage=xx, lastpage=yy,
  doi={10.5311/JOSIS.YYYY.II.NNN},
   received={December 24, 2015},
   returned={February 25, 2016},
   revised={July 13, 2016},
   accepted={September 5, 2016}, }

\newcommand{\mydoi}[1]{\href{http://dx.doi.org/#1}{doi:\protect\detokenize{#1}}}

%\renewcommand{\UrlLeft}{http:\sslash}
%\DeclareUrlCommand\myurl{\def\UrlLeft{}\def\UrlRight{}%
%\urlstyle{tt}}

\urlstyle{rm}
\makeatletter
% Inspired by http://anti.teamidiot.de/nei/2009/09/latex_url_slash_spacingkerning/
% but slightly less kern and shorter underscore
\let\UrlSpecialsOld\UrlSpecials
\def\UrlSpecials{\UrlSpecialsOld\do\/{\Url@slash}\do\_{\Url@underscore}}%
\def\Url@slash{\@ifnextchar/{\kern-.11em\mathchar47\kern-.2em}%
    {\kern-.0em\mathchar47\kern-.08em\penalty\UrlBigBreakPenalty}}
\def\Url@underscore{\nfss@text{\leavevmode \kern.06em\vbox{\hrule\@width.3em}}}
\makeatother

\hypersetup{
colorlinks=true,
linkcolor=black,
citecolor=black,
urlcolor=black
}

% Add the running author and running title information
\runningauthor{\begin{minipage}{.9\textwidth}\centering , \end{minipage}}
\runningtitle{Short Title for JOSIS Article}

% Document begins
\begin{document}
%\setcounter{page}{33}


% Insert your own title
\title{Long Title for JOSIS Article}

% Insert your manuscipts authors, affiliations, and addresses
  \author{Alice Anonymous}\affil{Some Institute of Technology}
  \author{Bob Security}\affil{Another University}

\maketitle

% Add 5-10 keywords for every submission
\keywords{add, five, to, ten, comma, separated, keywords}

% Add a short abstract of 150-250 words
\begin{abstract}
Add your article abstract here, 150-250 words.
\end{abstract}

\hypertarget{introduction}{%
\section{Introduction}\label{introduction}}

This template provides a guide to formatting articles for submission to
the Journal of Spatial Information Science, JOSIS,
\burl{http://josis.org}. When preparing an article for submission,
please follow this template closely, referring to past JOSIS articles
(open access on the JOSIS web site) for further examples.

\hypertarget{rmarkdown}{%
\section{RMarkdown}\label{rmarkdown}}

The template can handle bullet points:

\begin{itemize}
\tightlist
\item
  a
\item
  b
\end{itemize}

And tables (see Table @ref(tab:t1)):

\begin{Shaded}
\begin{Highlighting}[]
\NormalTok{knitr}\SpecialCharTok{::}\FunctionTok{kable}\NormalTok{(}\FunctionTok{head}\NormalTok{(mtcars), }\AttributeTok{caption =} \StringTok{"Test"}\NormalTok{, }\AttributeTok{booktabs =} \ConstantTok{TRUE}\NormalTok{)}
\end{Highlighting}
\end{Shaded}

\begin{longtable}[]{@{}lrrrrrrrrrrr@{}}
\caption{Test}\tabularnewline
\toprule
& mpg & cyl & disp & hp & drat & wt & qsec & vs & am & gear & carb \\
\midrule
\endfirsthead
\toprule
& mpg & cyl & disp & hp & drat & wt & qsec & vs & am & gear & carb \\
\midrule
\endhead
Mazda RX4 & 21.0 & 6 & 160 & 110 & 3.90 & 2.620 & 16.46 & 0 & 1 & 4 &
4 \\
Mazda RX4 Wag & 21.0 & 6 & 160 & 110 & 3.90 & 2.875 & 17.02 & 0 & 1 & 4
& 4 \\
Datsun 710 & 22.8 & 4 & 108 & 93 & 3.85 & 2.320 & 18.61 & 1 & 1 & 4 &
1 \\
Hornet 4 Drive & 21.4 & 6 & 258 & 110 & 3.08 & 3.215 & 19.44 & 1 & 0 & 3
& 1 \\
Hornet Sportabout & 18.7 & 8 & 360 & 175 & 3.15 & 3.440 & 17.02 & 0 & 0
& 3 & 2 \\
Valiant & 18.1 & 6 & 225 & 105 & 2.76 & 3.460 & 20.22 & 1 & 0 & 3 & 1 \\
\bottomrule
\end{longtable}

You can use concise citation syntax, e.g. (Champion et al. 2011).

\hypertarget{author-guidelines}{%
\section{Author guidelines}\label{author-guidelines}}

\subsection{Manuscript preparation}

Manuscripts must be written in English in a clear, direct, and active
style. All pages must be numbered sequentially. The manuscript should be
submitted as a PDF file based on this \LaTeX template.

\subsection{Title}

The title should be concise, and must not be more than 20 words. Authors
should also provide a short ``running title.'\,'

\subsection{Authors and institutional affiliations}

Authors are required to provide their full names and their institutional
affiliations, omitting postal addresses.

\subsection{Abstract}

The abstract should summarize the essential features of the article, and
must not exceed 250 words for full papers. Abbreviations should be
avoided in the abstract, and references should not be cited in the
abstract.

\subsection{Keywords}

Your submission must include between five and ten keywords for your
article. Accepted manuscripts must additionally specify further index
terms as appropriate.

\subsection{Main Text}

The main text should be divided into separate sections, and may be
further subdivided according to the areas to be discussed. The
manuscript style must be uniform throughout the text using 11pt Palatino
font. The first appearance of any abbreviations in the text should be
preceded by the full term, unless it is a standard abbreviation or unit
of measurement. Reference numbers should be given in square brackets in
the text. Common or assimilated words from Latin or other languages
should not be italicized, including per se, et al.

\subsection{Style}

Many examples of the journal style can be seen in existing JOSIS
published articles, \url{http://josis.org}. Please pay particular
attention to the following style requirements:

\begin{itemize}
\item Spelling: Please use standard American English spelling throughout.
\item Punctuation: JOSIS uses standard American English punctuation. In
particular, please ensure: 
\begin{itemize}
\item all lists items are always separated by
punctuation (e.g., ``a, b, and c'' but not ``a, b and c''); and 
\item commas and periods always appear inside quotation marks (e.g., ``x, y, and z.'' but not ``x, y, and z''.).
\end{itemize}
\item Capitalization: JOSIS style is to only use capitals only for the
beginning of sentences, proper nouns, names (e.g., Norman, ArcMap) and,
where appropriate, acronyms (e.g., GIS). Please avoid capitalization of
other words (e.g., ``geographic information systems (GIS)'' but not
``Geographic Information Systems (GIS)'') and in titles, including section headings (e.g., ``Affordance-based individuation of junctions in Open Street Map" but not ``Affordance-Based Individuation of Junctions in Open Street Map'').
\end{itemize}

\subsection{Figures}

Figures and Tables must be numbered consecutively with a concise
explanatory caption, and must be referred to in the main text with
capitalized reference (e.g.,
\texttt{Figure\ \textbackslash{}ref\{fig:1\}\textquotesingle{}\textquotesingle{}\ or}Table
\ref{tab:1}'\,'). Figures and Tables must appear in the text close to
where they are first referred to in the main text. Figure and table
captions come below the figure or table. Do not collect figures or
tables together at the end of the article. Authors of accepted articles
will need to supply high quality versions of all figures as separate
.eps (encapsulated postscript) files.

\begin{Shaded}
\begin{Highlighting}[]
\FunctionTok{plot}\NormalTok{(}\DecValTok{1}\SpecialCharTok{:}\DecValTok{9}\NormalTok{, (}\DecValTok{1}\SpecialCharTok{:}\DecValTok{9}\NormalTok{)}\SpecialCharTok{\^{}}\DecValTok{2}\NormalTok{)}
\end{Highlighting}
\end{Shaded}

\begin{figure}
\centering
\includegraphics{Untitled_files/figure-latex/fig1-1.pdf}
\caption{Illustrative figure.}
\end{figure}

\begin{table}
\centering
\begin{tabular}{lrr}
\hline
Text column & Numerical column 1 & Numerical column 2\\
\hline
First row& 10 & 0.003\\
Second row& 52& 10.037\\
Third row& 729 & 150.315\\
...& ...& ...\\
\hline
\end{tabular}
\caption{Example table with preferred line rules and alignment.}\label{tab:1}
\end{table}

\subsection{Algorithms}

Algorithms should be formatted using standard algorithm packages where
possible, as in Algorithm\textasciitilde{}\ref{alg:1}.

\begin{algorithm}[htb]
\caption{Example algorithm formatting after \cite{zhong16.IJGIS}} \label{alg:1}
\footnotesize
\begin{algorithmic} [1]                   % enter the algorithmic environment
\Require A finite set of two-dimensional points $P \subset \mathbb{R} \times \mathbb{R}$ and one parameter $\lambda \in \mathbb{R}$
\State Construct the Delaunay triangulation $DT(P)$ of $P$
\State $\Delta \gets DT(P)$
\State Construct the list $B$ of exterior edges of $DT(P)$
\State Sort the list $B$ in descending order of edge length
\State Initialize the $v\!-\!boundary$ function
\State Set the root ($r$) of $\mathbf{T}_{{\chi}}(P,\, \lambda)$ to be \{edge = $\emptyset$, oppositeVertex = $\emptyset$, length = $\emptyset$\} 
\State Construct the list of parent nodes ($PN$) for the elements in $B$
\State Set each element in $PN$ to be $r$
\State $O(P,\, \lambda) \gets \emptyset$
\While{$B$ is not empty} 
    \State $e = \{d_1,\, d_2\} \gets$ pop($B$)
    \State $p \gets$ pop($PN$)
    \State $o \gets$ opposite vertex of $e$ in $\Delta$
    \State $N \gets$ \{edge = $e$, oppositeVertex = $o$, length = $||e||$\}
    \State Insert $N$ in $\mathbf{T}_{{\chi}}(P,\, \lambda)$ as a child of $p$
    \State $node(d_1) \gets N$
    \State $node(d_2) \gets N$
    \State Append $N$ to $O(P,\, \lambda)$
    \If{$||e|| > \lambda$ \textbf{and} $v\!-\!boundary(o) = false$}
        \State Remove $e$ from $\Delta$
        \State $v\!-\!boundary(o) = true$
        \State Insert the arms of $e$ in $\Delta$ into $B$ in order of edge length
        \State Insert $N$ into $PN$ at the corresponding position of the arms of $e$ in $B$ 
    \EndIf
\EndWhile
\State \textbf{Return} $\chi(P,\, \lambda)$ formed by leaves of $\mathbf{T}_{{\chi}}(P,\, \lambda)$, $DT(P)$, $\mathbf{T}_{{\chi}}(P,\, \lambda)$ and $O(P,\, \lambda)$
\end{algorithmic}
\end{algorithm}

\subsection{Footnotes}

Footnotes are strongly discouraged in text. Where footnotes must be
used, they should be numbered consecutively.

\hypertarget{references}{%
\section{References}\label{references}}

References must be listed in the numerical system (ACM). Citations must
be numbered sequentially {[}in square brackets{]} in the main text. Full
numbered references must be listed in the reference section in
alphabetical order. The reference numbers must be finalized and the
bibliography must be fully formatted before submission. Examples of
citation styles included in the bibliography for this document include
journal articles \cite{overEtAl2010,arkin}, authored books
\cite{bailey}, edited books \cite{miller09.BOOK}, articles in
proceedings \cite{champion11}, articles in books or collections
\cite{Grosso12}, theses \cite{ruas99}, technical reports
\cite{blasertr}, and web resources \cite{web}.

\subsection{DOIs}

All JOSIS articles must list the DOI of all references, where a DOI
exists, see
\url{http://josis.org/index.php/josis/about/submissions#authorGuidelines}
for examples. Please check carefully to add the DOIs for cited
references, adding DOIs to all references that have one. DOIs may be
added in the ``doi'\,' field of the bibtex file.

DOIs can be found via \url{http://www.crossref.org/guestquery/} as well
as many other search engines and publisher pages (e.g., Scopus,
SpringerLink).

\section{About JOSIS}

The Journal of Spatial Information Science (JOSIS) is an international,
interdisciplinary, open-access journal dedicated to publishing
high-quality, original research articles in spatial information science.
The journal aims to publish research spanning the theoretical
foundations of spatial and geographical information science, through
computation with geospatial information, to technologies for
geographical information use.

JOSIS encourages submissions from topics including, but not limited to
spatial and spatiotemporal information systems; computational geometry,
geocomputation, spatial algorithms; geovisualization, cartography, and
geographical user interfaces; computing with spatiotemporal information
under uncertainty; spatial cognition and qualitative spatial reasoning;
spatial data models and structures; conceptual models of space and
geoontology; distributed and parallel spatial computing, web-based GIS,
and interoperability; context- and location-aware computing; and
applications to GIS, spatial databases, location-based services,
geosensor networks, and geosensor web. The journal publishes full-length
original research articles, as well as survey-style review papers. In
addition, the journal publishes shorter articles in three sections:
reports from community activities, letters to the editors, and book
reviews.

\section*{Acknowledgments}

Acknowledgments appear in a separate unnumbered section before the
bibliography. Please acknowledge anyone (individual/company/institution)
who has contributed to the study, including substantial contributions to
the conception, design, acquisition of data; analysis and interpretation
of data; drafting the manuscript; or provided critical comments
resulting in revisions to content. For each author, please list the
source(s) of any funding or financial contributions related to the
study.

\hypertarget{refs}{}
\begin{CSLReferences}{1}{0}
\leavevmode\vadjust pre{\hypertarget{ref-champion11}{}}%
Champion, Nicolas, Didier Boldo, Marc Pierrot-Deseilligny, and Georges
Stamon. 2011. {``{2d} Change Detection from Satellite Imagery:
Performance Analysis and Impact of the Spatial Resolution of Input
Images.''} In \emph{Proc.~IEEE International Geoscience and Remote
Sensing Symposium (IGARSS)}, 1421--24.
\url{https://doi.org//mydoi\%7B10.1109/igarss.2011.6049332\%7D}.

\end{CSLReferences}

\bibliographystyle{josisacm}
\bibliography{josisexample.bib}

\end{document}
